% Conclusion
\label{sec:conclusion}

\subsection{Summary}

We presented \systemname{}, a real-robot navigation system that bridges the
sim-to-real gap for attribute-aware language grounding. By making attributes
explicit, confidence-weighted, and hierarchically verified, we achieve
\textbf{72.2\%} subgoal success and \textbf{54.3\%} full-chain completion on
real-world multi-step instructions—\textbf{+23.6\%} over VLMaps and
\textbf{+22.9\%} full-chain over CapNav~\cite{capnav2026}.

\textbf{Key contributions}:
\begin{enumerate}
    \item First real-robot system with explicit attribute verification and
    per-attribute confidence estimation

    \item Hierarchical verification cascade improving long-horizon navigation
    from 20\% to 54.3\%

    \item Adaptive threshold calibration eliminating manual tuning

    \item Online map updates with 89.2\% recovery from moved objects

    \item Comprehensive evaluation (7 real + 20 sim scenes, 4 baselines, 5
    ablations) addressing limitations in prior work
\end{enumerate}

\subsection{Limitations}

\textbf{L1. Transparent/reflective objects}: Depth sensors fail on glass,
mirrors. Could be addressed with multi-modal fusion (tactile, audio).

\textbf{L2. Heavy occlusions}: Objects <20\% visible are rejected. Active
exploration~\cite{chaplot2020active} could explicitly search for occluded targets.

\textbf{L3. Temporal attributes}: \textit{``The mug I left here yesterday''}
requires memory beyond static maps. Could use persistent object IDs with
temporal tracking.

\textbf{L4. Ambiguity resolution}: When two objects match equally (\textit{two
red mugs}), the system picks one arbitrarily. Could ask user for clarification.

\textbf{L5. Compute requirements}: Jetson AGX Orin (64GB, \$2000) is expensive.
Model quantization and pruning could enable deployment on cheaper hardware.

\subsection{Future Work}

\textbf{Active exploration}: Instead of passively rejecting low-confidence
objects, actively move to acquire better views or disambiguate candidates.

\textbf{Multi-robot collaboration}: Multiple robots share semantic maps and
attribute observations, improving coverage and confidence estimates.

\textbf{Manipulation integration}: Extend navigation to \textit{``pick up the
red mug on the wooden table''} by adding grasp pose estimation to attribute
verification.

\textbf{Long-term autonomy}: Track object movements over days/weeks, learn
temporal patterns (\textit{``the mug is usually on the kitchen counter in the
morning''}), and predict locations.

\textbf{Open-set spatial relations}: Current system handles simple relations
(\textit{on, under, next to}). Could learn compositional spatial
grounding~\cite{kuo2024spatial} for complex descriptions (\textit{``between the
chair and window, but closer to the chair''}).

\subsection{Broader Impact}

Attribute-aware navigation enables service robots to follow human-like
descriptions in homes, hospitals, and warehouses. Potential applications:

\begin{itemize}
    \item \textbf{Assistive robotics}: \textit{``Fetch my red medication bottle
    from the bathroom counter''}

    \item \textbf{Warehouse logistics}: \textit{``Retrieve the blue box on the
    top shelf, third from the left''}

    \item \textbf{Healthcare}: \textit{``Deliver the IV bag on the rolling cart
    near the window to Room 302''}
\end{itemize}

\textbf{Ethical considerations}: Attribute-based queries (e.g., \textit{``person
with red shirt''}) could enable surveillance. Future work should incorporate
privacy-preserving constraints and user consent mechanisms.

\subsection{Conclusion}

By treating attributes as first-class, verifiable hypotheses rather than
implicit embeddings, \systemname{} demonstrates that fine-grained language
grounding is achievable on real robots in diverse, dynamic environments. We hope
this work inspires future research on uncertainty-aware perception, hierarchical
verification, and human-aligned navigation.
